%                                                                 aa.dem
% AA vers. 8.2, LaTeX class for Astronomy & Astrophysics
% demonstration file
%                                                       (c) EDP Sciences
%-----------------------------------------------------------------------
%
%\documentclass[referee]{aa} % for a referee version
%\documentclass[onecolumn]{aa} % for a paper on 1 column  
%\documentclass[longauth]{aa} % for the long lists of affiliations 
%\documentclass[rnote]{aa} % for the research notes
%\documentclass[letter]{aa} % for the letters 
%\documentclass[bibyear]{aa} % if the references are not structured 
% according to the author-year natbib style

%
\documentclass{aa}  

%
\usepackage{graphicx}
%%%%%%%%%%%%%%%%%%%%%%%%%%%%%%%%%%%%%%%%
\usepackage{txfonts}
%%%%%%%%%%%%%%%%%%%%%%%%%%%%%%%%%%%%%%%%
\usepackage{hyperref}
% To add links in your PDF file, use the package "hyperref"
% with options according to your LaTeX or PDFLaTeX drivers.
%%%%%%%%%%%%%%%%%%%%%%%%%%%%%%%%%%%%%%%%
\usepackage{natbib}
%%%%%%%%%%%%%%%%%%%%%%%%%%%%%%%%%%%%%%%%
\usepackage[pyfuture=none]{pythontex}

\usepackage[T1]{fontenc}
\usepackage[utf8]{inputenc} 


\newcommand{\pytex}{Python\TeX}

%
\begin{document} 


   \title{Python\TeX}

   \subtitle{}

   \author{João Faria
          \inst{1}
          }

   \institute{Programmer's Club, Feb 26 2015}

   \date{Received ...; accepted ....}

% \abstract{}{}{}{}{} 
% 5 {} token are mandatory
 
  \abstract
  % context heading (optional)
  % {} leave it empty if necessary  
   {To investigate the physical nature of the `nuc\-leated instability' of
   proto giant planets, the stability of layers
   in static, radiative gas spheres is analysed on the basis of Baker's
   standard one-zone model.}
  % aims heading (mandatory)
   {It is shown that stability
   depends only upon the equations of state, the opacities and the local
   thermodynamic state in the layer. Stability and instability can
   therefore be expressed in the form of stability equations of state
   which are universal for a given composition.}
  % methods heading (mandatory)
   {The stability equations of state are
   calculated for solar composition and are displayed in the domain
   $-14 \leq \lg \rho / \mathrm{[g\, cm^{-3}]} \leq 0 $,
   $ 8.8 \leq \lg e / \mathrm{[erg\, g^{-1}]} \leq 17.7$. These displays
   may be
   used to determine the one-zone stability of layers in stellar
   or planetary structure models by directly reading off the value of
   the stability equations for the thermodynamic state of these layers,
   specified
   by state quantities as density $\rho$, temperature $T$ or
   specific internal energy $e$.
   Regions of instability in the $(\rho,e)$-plane are described
   and related to the underlying microphysical processes.}
  % results heading (mandatory)
   {Vibrational instability is found to be a common phenomenon
   at temperatures lower than the second He ionisation
   zone. The $\kappa$-mechanism is widespread under `cool'
   conditions.}
  % conclusions heading (optional), leave it empty if necessary 
   {}

   \keywords{giant planet formation --
                $\kappa$-mechanism --
                stability of gas spheres
               }

   \maketitle
%
%________________________________________________________________

\section{Introduction}

  This paper is about \pytex, which allows to enter Python code within a \LaTeX document, execute the code and access its output in the document.

  Let us try a few simple commands:

  \begin{itemize}
    \item \pygment{latex}{\py} returns a string representation of its argument. \\ For example, \pygment{latex}{\py{2 + 4**2}} produces \py{2 + 4**2}, and \pygment{latex}{\py{'ABC'.lower()}} produces \py{'ABC'.lower()}.
    \\

    \item \pygment{latex}{\pyc} executes code. \\
    For example, \pygment{latex}{\pyc{var = 2}} \pyc{var = 2} creates a variable, and then its value can be accessed later via \pygment{latex}{\py{var}}: \py{var}.\\
    Another example: \pygment{latex}{\pyc{from numpy import pi}} \pyc{from numpy import pi} and later in the document \pygment{latex}{\py{pi}}: \py{pi}
    \\

    \item \pygment{latex}{\pyb} executes and typesets code.\\
    For example, \pygment{latex}{\pyb{var = 2}} typesets \pyb{var = 2} in addition to creating the variable.
  \end{itemize}

  There are also a handful of environments

  \begin{itemize}
    \item \pygment{latex}{\begin{pycode}} code that is executed but not typeset
    \item \pygment{latex}{\begin{pyverbatim}} code that is typeset but not executed
    \item \pygment{latex}{\begin{pyblock}} code that is both executed and typeset
  \end{itemize}

  All commands and environments can be prepended with \pygment{latex}{\pylab} to import matplotlib's \textt{pylab} module, or with \pygment{latex}{\sympy} to import \textt{sympy}.
  


%__________________________________________________________________

\section{Why is this useful?}

  \subsection{To make tables automatically}

    Table values can be filled in from Python and the \pygment{latex}{\pycode} environment used to include them in the document.

\begin{pycode}
print r'\begin{table}[h]'
print r'\begin{tabular}{l|lll}'
a, b, c = 12, 3, 6
print r'name & a & b & c \\'
print 'value & {} & {} & {}'.format(a,b,c)
print r'\end{tabular}'
print r'\end{table}'
\end{pycode}


  \subsection{To avoid copy/paste}

    Copying and pasting values into the \LaTeX\ document can be tedious and error-prone. If the variables are set in Python, we can access them by their name.

\begin{pygments}{text}
\begin{pycode}
import numpy as np
stars = np.loadtxt('sample.rdb', dtype='S')
nstars = d.size
nstars_HD = len([s for s in stars if s.startswith('HD')])
\end{pycode}
\end{pygments}

\begin{pycode}
import numpy as np
d = np.loadtxt('sample.rdb', dtype='S')
nstars = d.size
nstars_HD = len([star for star in d if star.startswith('HD')])
\end{pycode}

    There are \pygment{latex}{\py{nstars}}:\py{nstars} stars in our sample, \pygment{latex}{\py{nstars_HD}}:\py{nstars_HD} of which start with `HD'.


  \subsection{For automatic calculus}

    If we use SymPy we can have Python perform algebraic manipulations and then properly typeset the results.

\begin{sympyblock}
var('x, y, z')
z = x + y
\end{sympyblock}

To typeset the variable $z$ in an equation, use \pygment{latex}{\sympy{z}}. 
We set $z$ to $\sympy{z}$.

Calculus becomes easy(ier):

\begin{sympyblock}
f = x**3 + cos(x)**5
g = Integral(f, x)
\end{sympyblock}

The LHS of Eq. \ref{eq:sympy} was typeset with \pygment{latex}{\sympy{g}}. The integral is then evaluated and typeset with \pygment{latex}{\sympy{g.doit()}}

\begin{equation} \label{eq:sympy}
  \sympy{g}=\sympy{g.doit()}  
\end{equation}


It's easy to use arbitrary symbols in equations.

\begin{sympyblock}
phi = Symbol(r'\phi')
h = Integral(exp(-phi**2), (phi, 0, oo))
\end{sympyblock}

\begin{equation} 
  \sympy{h}=\sympy{h.doit()}
\end{equation}

  \subsection{To make reproducible papers}

    If the final version of your paper is compiled with \pytex, you know for sure that the numerical results included in it are the exact ones coming out of your code.

    Therefore, if you happen to publish your code, and someone tries to reproduce your paper, they will get the exact same answer.


  \subsection{To include figures}

    It's useful to have the code that created a figure, together with that figure.

\begin{pylabblock}
rc('text', usetex=True)
rc('font', family='serif', size=10)
rc('legend', fontsize=10.0)
x = linspace(0, 10)
figure(figsize=(4, 2.5))
plot(x, sin(x), label='$\sin(x)$')
xlabel(r'$x\mathrm{-axis}$')
ylabel(r'$y\mathrm{-axis}$')
legend(loc=4)
savefig('myplot.pdf', bbox_inches='tight')
\end{pylabblock}

    Now we just have to load the \texttt{myplot.pdf} file and show it in Fig. \ref{fig:sine}.
    
    \begin{figure}
    \centering
    \includegraphics[width=\hsize]{myplot.pdf}
      \caption{A sine function.}
      \label{fig:sine}
    \end{figure}

    We could also create the figure in a separate file \texttt{makefigs.py} and assign it to a variable \texttt{fig1} and then use

\begin{pygments}{python}
\begin{pylabcode}
execfile('makefigs.py')
fig1.savefig('fig1.pdf', bbox_inches='tight')
\end{pylabcode}
\end{pygments}

\begin{pylabcode}
pytex.add_dependencies('makefigs.py')
execfile('makefigs.py')
fig1.savefig('fig1.pdf', bbox_inches='tight')
pytex.add_created(['fig1.pdf'])
\end{pylabcode}


    \begin{figure}
    \centering
    \includegraphics[width=\hsize]{fig1.pdf}
      \caption{A cosine function.}
      \label{fig:cosine}
    \end{figure}

\section{dePython\TeX}

  Journals don't really like \LaTeX\ files with weird commands in them. So there is a {\tt depythontex} command that can convert a Python\TeX\ document into a plain \LaTeX\ document. This does not change anything that was included from Python.



%-------------------------------------------------------------------

\begin{thebibliography}{}

\bibitem{poore2013}
  Geoffrey M. Poore,
  \emph{Reproducible Documents with PythonTeX}.
  Proceedings of the 12th Python in Science Conference, 78-84, 2013,
  \href{http://conference.scipy.org/proceedings/scipy2013/poore.html}{http://conference.scipy.org/proceedings/scipy2013/poore.html}

\end{thebibliography}

\end{document}